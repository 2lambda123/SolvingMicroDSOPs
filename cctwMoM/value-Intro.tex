
  Although section~\ref{subsec:vVsuP} argued that our problem is more
  efficiently solved by constructing the consumption rule than by
  approximating the value function, often it is useful to know the
  value function as well as the consumption rule.  Fortunately, many
  of the tricks used when solving for the consumption rule have a
  direct analogue in approximation of the value function.

  Consider the perfect foresight (or ``optimist's'') problem in period $T-1$:
  \begin{eqnarray*}
    \bar{\vFunc}_{T-1}({m}_{T-1}) & \equiv & \util(\cRat_{T-1})+\Discount \util(\cRat_{T})
    \\ & = & \util(\cRat_{T-1})\left(1+\Discount ((\Discount_{T}\Rfree)^{1/\CRRA})^{1-\CRRA}\right)
    \\ & = & \util(\cRat_{T-1})\left(1+\Discount (\Discount_{T}\Rfree)^{1/\CRRA-1}\right)
    \\ & = & \util(\cRat_{T-1})\left(1+(\Discount_{T}\Rfree)^{1/\CRRA}/\Rfree\right)
    \\ & = & \util(\cRat_{T-1})\underbrace{\mbox{PDV}_{t}^{T}(\cRat)/\cRat_{T-1}}_{\equiv \mathbb{C}_{t}^{T}}
  \end{eqnarray*}
  where $\mathbb{C}_{t}^{T}=\mbox{PDV}_{t}^{T}(\cRat)$ is the present discounted value of consumption.
  A similar function can be constructed recursively for earlier periods, yielding
  the general expression \hypertarget{vFuncPF}{}
