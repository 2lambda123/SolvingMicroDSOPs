
  Often it is useful to know the value function as well as the consumption rule.  Fortunately, many of the tricks used when solving for the consumption rule have a direct analogue in approximation of the value function.

  Consider the perfect foresight (or ``optimist's'') problem in period $T-1$:
  \begin{equation*}\begin{gathered}\begin{aligned}
        \bar{\vFunc}_{T-1}({m}_{T-1})  & \equiv  \uFunc(\cNrm_{T-1})+\DiscFac \uFunc(\cNrm_{T})
        \\  & = \uFunc(\cNrm_{T-1})\left(1+\DiscFac ((\DiscFac_{T}\Rfree)^{1/\CRRA})^{1-\CRRA}\right)
        \\  & = \uFunc(\cNrm_{T-1})\left(1+\DiscFac (\DiscFac_{T}\Rfree)^{1/\CRRA-1}\right)
        \\  & = \uFunc(\cNrm_{T-1})\left(1+(\DiscFac_{T}\Rfree)^{1/\CRRA}/\Rfree\right)
        \\  & = \uFunc(\cNrm_{T-1})\underbrace{\mbox{PDV}_{t}^{T}(\cNrm)/\cNrm_{T-1}}_{\equiv \mathbb{C}_{t}^{T}}
      \end{aligned}\end{gathered}\end{equation*}
  where $\mathbb{C}_{t}^{T}=\mbox{PDV}_{t}^{T}(\cNrm)$ is the present discounted value of consumption.
  A similar function can be constructed recursively for earlier periods, yielding
  the general expression \hypertarget{vFuncPF}{}
