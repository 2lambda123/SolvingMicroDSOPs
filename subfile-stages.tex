  Although our focus so far has been on the consumer's consumption problem once $\mNrm$ is known, for articulating the steps of the computational solution it will be useful to distinguish a sequence of three steps within each time period (we use the word `steps' to capture the proposition that we are not really thinking of these as occurring at different moments in time -- only that we are ordering the things that happen within a given moment into an ordered sequence).  The first step captures calculations that need to be performed before the shocks are realized, the middle step is after shocks have been realized but before the consumption decision has been made (this corresponds to the timing of $\vFunc$ in the treatment above), and the final step captures the situation \textit{after} the consumption decision has been made.

We need to define notation for these three steps. We will use $\BegStget$ as a marker for the step in period $t$ before shocks have been realized, and the $\EndStget$ as the indicator for the situation once the choice has been made, implicitly leaving the unmarked $t$ to indicate the step in which the decision is made.
